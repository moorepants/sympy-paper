
SymPy supports construction and manipulation of boolean expressions
through the \texttt{logic} module. SymPy symbols can be used as
propositional variables and also be substituted as \texttt{True}
or \texttt{False}. A good number of manipulation features for boolean
expressions have been implemented in the \texttt{logic} module.

\subsubsection{Constructing boolean expressions}

A boolean variable can be declared as a SymPy symbol. Python
operators \&, \textbar{} and \textasciitilde{} are overloaded for logical \texttt{And},
\texttt{Or} and \texttt{negate}. Several others like \texttt{Xor},
\texttt{Implies} can be constructed with \^{}, \textgreater\textgreater{} respectively.
The above are just a shorthand, expressions can also be constructed
by directly calling \verb|And()|, \verb|Or()|, \verb|Not()|,
\verb|Xor()|, \verb|Nand()|, \verb|Nor()|, etc.

\begin{verbatim}
>>> from sympy import *
>>> x, y, z = symbols('x y z')
>>> e = (x & y) | z
>>> e.subs({x: True, y: True, z: False})
True
\end{verbatim}

\subsubsection{CNF and DNF}

Any boolean expression can be converted to conjunctive normal
form, disjunctive normal form and negation normal form. The
API also permits to check if a boolean expression is in any
of the above mentioned forms.

\begin{verbatim}
>>> from sympy import *
>>> x, y, z = symbols('x y z')
>>> to_cnf((x & y) | z)
And(Or(x, z), Or(y, z))
>>> to_dnf(x & (y | z))
Or(And(x, y), And(x, z))
>>> is_cnf((x | y) & z)
True
>>> is_dnf((x & y) | z)
True
\end{verbatim}

\subsubsection{Simplification and Equivalence}

The module supports simplification of given boolean expression
by making deductions on it. Equivalence of two expressions can
also be checked. If so, it is possible to return the mapping of
variables of two expressions so as to represent the
same logical behaviour.

\begin{verbatim}
>>> from sympy import *
>>> a, b, c, x, y, z = symbols('a b c x y z')
>>> e = a & (~a | ~b) & (a | c)
>>> simplify(e)
And(Not(b), a)
>>> e1 = a & (b | c)
>>> e2 = (x & y) | (x & z)
>>> bool_map(e1, e2)
(And(Or(b, c), a), {b: y, a: x, c: z})
\end{verbatim}

\subsubsection{SAT solving}

The module also supports satisfiability checking of a given
boolean expression. If satisfiable, it is possible to return
a model for which the expression is satisfiable. The API also
supports returning all possible models. The SAT solver has
a clause learning DPLL algorithm implemented with watch
literal scheme and VSIDS heuristic\cite{moskewicz2008method}.

\begin{verbatim}
>>> from sympy import *
>>> a, b, c = symbols('a b c')
>>> satisfiable(a & (~a | b) & (~b | c) & ~c)
False
>>> satisfiable(a & (~a | b) & (~b | c) & c)
{b: True, a: True, c: True}
\end{verbatim}
