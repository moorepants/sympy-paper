%% Sets

SymPy supports representation of a wide variety of sets, this is achieved by
first defining abstract representation for a smaller number of atomic set
classes and then combining and transforming them using various set operations.

Each of the set classes inherits from the base set class and defines rules to
check membership of a SymPy object in that set, to calculate union,
intersection and set difference. In cases we are not able to evaluate these
operations to atomic set classes they are represented as abstract unevaluated
objects.


We have the following atomic set classes in SymPy.

\begin{itemize}

    \item \verb|EmptySet|: represents the empty set $\emptyset$.

    \item \verb|UniversalSet|: Everything is a member of Universal Set.
        Union of Universal Set with any set gives Universal Set and
        intersection leads to the other set itself.

    \item \verb|FiniteSet| is functionally equivalent to python's set
        object. Its members can be any SymPy object including other sets
        themselves.

    \item \verb|Integers| represents set of Integers $\mathbb{Z}$.

    \item \verb|Naturals| represents set of Natural numbers $\mathbb{N}$ i.e.,
        set of positive integers.

    \item \verb|Naturals0| represents the whole numbers which are all the
        non-negative integers, inclusive of zero.

    \item \verb|Range| represents a range of integers and is defined by
        specifying a start value, an end value and a step size. Range is
        functionally equivalent to python's range except the fact that it
        accepts infinity at end points allowing us to represent infinite
        ranges.


    \item \verb|RealInterval| is specified by giving the start and end point
        and specifying if it is open or closed in the respective ends. The set
        of real numbers is represented as a special case of a real interval
        where the start point is negative infinite and the end point is
        positive infinite.


\end{itemize}


%% Operations

Other than unevaluated classes of Union, Intersection and Set Difference
operations, we have following set classes.

\begin{itemize}

    \item \verb|ProductSet| abstractly defines the Cartesian product of two
        or more sets. Product Set is useful when representing higher
        dimensional spaces. For example to represent a three dimensional space
        we simply take the Cartesian product of three Real sets.

    \item \verb|ImageSet| represents the image of a function when applied to a
        particular set. In notation Image Set of a function $F$ w.r.t a set $S$
        is $\{ F(x) | x \in S \}$ In particular we use Image Set to represent
        the set of infinite solutions from trigonometric equations.


    \item \verb|ConditionSet| represents subset of a set who's members
        satisfies a particular condition. In notation Condition Set of set $S$
        w.r.t to a condition $H$ is $\{x | H(x), x \in S \}$. We use Condition Set
        to represent the set of solutions of an equation or an inequality where
        the equation or the inequality is the condition and the set is the
        domain in which we aim to find the solution.


\end{itemize}

A few other classes are implemented as special cases of the classes described
above. The real number \verb|Reals| is implemented as a special case of real
interval where the start point is negative infinity and the end point is
positive infinity. \verb|ComplexRegion| is implemented as a special case of
\verb|ImageSet|, \verb|ComplexRegion| supports both polar and rectangular
representation of region on the complex plane.
