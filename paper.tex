% SIAM Article Template
\documentclass[review]{siamart0216}
\usepackage{hyperref}
\usepackage{graphicx}
\usepackage{amsmath}
\usepackage{caption}
\graphicspath{ {images/} }

\usepackage{amsmath}
\usepackage{url}
\usepackage{hyperref}

% for nice source code syntax highlighting, also provides Listing env
\usepackage{minted}

% this is required for all the \url{} commands in the bib file
%\usepackage{hyperref}

% for nice units
\usepackage{siunitx}

% for images: png, pdf, etc
\usepackage{graphicx}

% for nice table formatting, i.e., /toprule, /midrule, etc
\usepackage{booktabs}

% to allow for \verb++ declarations in captions.
\usepackage{cprotect}

% to allow usage of \mathbb symbols
\usepackage{amssymb}

\usepackage{longtable}

\title{SymPy: Symbolic Computing in Python}

\input{authors}

\begin{document}
\maketitle

\begin{abstract}
  Computer algebra systems (CAS) play an important role in scientific computing
  with the earliest packages dating back into the 1960s.
  Numerous open and proprietary software packages exist that aid users in
  complex algebra, calculus, and other mathematical concepts that drive and
  support both symbolic and numerical computing needs.
  Here we present a CAS written in the Python programming language that is
  developed by a large team of scientists, engineers, and software developers
  in a bazaar-style open source development model.
  The accessibility of the codebase and the community model allow SymPy to
  rapidly respond to the needs of the community of users without the knowledge
  or consent of the original developers.
  This paper gives an overview of what SymPy is, what it offers with respect to
  similar software, and how it fits into the scientific python ecosystem.
  SymPy's core architecture is highlighted to show how it provides a
  foundation for good design patterns and efficient, readable code.
  To close, we present why and how SymPy is useful to researchers, educators,
  and the scientific software industry with examples of its use in these
  fields.
  The supplementary materials further outline the fine details of the
  architecture, various design decisions, and algorithms that make SymPy what
  it is today.
\end{abstract}

\section{Introduction}

%% What sympy is, where to download etc.
%%
%% List other major CASs.
%%
%% Why SymPy.

\input{introduction.tex}

\section{Architecture}


%% I volunteer to write this section. --Aaron
%%
%% Representing symbolic expressions using Python objects

\subsection{The Core}

Every symbolic expression in SymPy is an instance of a Python class.
Expressions are represented by expression trees. The operators are represented
by the type of an expression and the child nodes are stored in the
\texttt{args} attribute. A leaf node in the expression tree has an empty
\texttt{args}.
The \texttt{args} attribute is provided by the class \texttt{Basic},
which is a superclass of all SymPy objects and
provides common methods to all SymPy tree-elements.
For example, take the expression $xy + 2$.

\begin{verbatim}
>>> from sympy import *
>>> x, y = symbols('x y')
>>> expr = x*y + 2
\end{verbatim}

The expression \texttt{expr} is an addition, so it is of type \texttt{Add}. The child
nodes of \texttt{expr} are \texttt{x*y} and \texttt{2}.

\begin{verbatim}
>>> type(expr)
<class 'sympy.core.add.Add'>
>>> expr.args
(2, x*y)
\end{verbatim}

We can dig further into the expression tree to see the full expression. For
example, the first child node, given by \texttt{expr.args[0]} is
\texttt{2}. Its class is \texttt{Integer}, and it has empty \texttt{args},
indicating that it is a leaf node.

\begin{verbatim}
>>> expr.args[0]
2
>>> type(expr.args[0])
<class 'sympy.core.numbers.Integer'>
>>> expr.args[0].args
()
\end{verbatim}

The function \texttt{srepr} gives a string representing a valid Python code,
containing all the nested class constructor calls to create the given expression.

\begin{verbatim}
>>> srepr(expr)
"Add(Mul(Symbol('x'), Symbol('y')), Integer(2))"
\end{verbatim}

Every SymPy expression satisfies a key invariant, namely,
\verb|expr.func(*expr.args) == expr|. This means that expressions are
rebuildable from their \texttt{args}~\footnote{\texttt{expr.func} is used
  instead of \texttt{type(expr)} to allow the function of an expression to be
  distinct from its actual Python class. In most cases the two are the same.}.
Here, we note that in SymPy, the \texttt{==} operator represents exact
structural equality, not mathematical equality. This allows one to test if any
two expressions are equal to one another as expression trees.

Python allows classes to overload operators. The Python interpreter translates
the above \texttt{x*y + 2} to, roughly,
\verb|(x.__mul__(y)).__add__(2)|. \texttt{x} and \texttt{y}, returned from
the \texttt{symbols} function, are \texttt{Symbol} instances. The \texttt{2}
in the expression is processed by Python as a literal, and is stored as
Python's builtin \texttt{int} type. When \texttt{2} is called by the
\verb|__add__| method, it is converted to the SymPy type \verb|Integer(2)|. In
this way, SymPy expressions can be built in the natural way using Python
operators and numeric literals.

One must be careful in one particular instance. Python does not have a builtin
rational literal type. Given a fraction of integers such as \texttt{1/2},
Python will perform floating point division and produce
\texttt{0.5}~\footnote{This is the behavior in Python 3. In Python 2,
  \texttt{1/2} will perform integer division and produce \texttt{0}, unless
  one uses \texttt{from \_\_future\_\_ import division}.}. Python uses eager
evaluation, so expressions like \texttt{x + 1/2} will produce \texttt{x +
  0.5}, and by the time any SymPy function sees the \texttt{1/2} it has
already been converted to \texttt{0.5} by Python. However, for a CAS like
SymPy, one typically wants to work with exact rational numbers whenever
possible. Working around this is simple, however:  one can wrap one of the
integers with \texttt{Integer}, like \verb|x + Integer(1)/2|, or using
\verb|x + Rational(1, 2)|. SymPy provides a function \texttt{S} which can be
used to convert objects to SymPy types with minimal typing, such as \verb|x + S(1)/2|.
This gotcha is a small downside to using Python directly instead
of a custom domain specific language (DSL), and we consider it to be worth it
for the advantages listed above.

%%
%% Assumptions
\subsection{Assumptions}

An important feature of the SymPy core is the assumptions system. The
assumptions system allows users to specify that symbols have certain common
mathematical properties, such as being positive, imaginary, or integer. SymPy
is careful to never perform simplifications on an expression unless the
assumptions allow them. For instance, the identity $\sqrt{x^2} = x$ holds if
$x$ is nonnegative ($x\ge 0$). If $x$ is real, the identity $\sqrt{x^2}=|x|$
holds. However, for general complex $x$, no such identity holds.

By default, SymPy performs all calculations assuming that variables are
complex valued. This assumption makes it easier to treat mathematical problems
in full generality.

\begin{verbatim}
>>> x = Symbol('x')
>>> sqrt(x**2)
sqrt(x**2)
\end{verbatim}

By assuming symbols are complex by default, SymPy avoids performing
mathematically invalid operations. However, in many cases users will wish to
simplify expressions containing terms like $\sqrt{x^2}$.

Assumptions are set on \texttt{Symbol} objects when they are created. For
instance \verb|Symbol('x', positive=True)| will create a symbol named
\texttt{x} that is assumed to be positive.

\begin{verbatim}
>>> x = Symbol('x', positive=True)
>>> sqrt(x**2)
x
\end{verbatim}

Some common assumptions that SymPy allows are \texttt{positive},
\texttt{negative}, \texttt{real}, \texttt{nonpositive}, \texttt{nonnegative},
\texttt{real}, \texttt{integer}, and \texttt{commutative}~\footnote{If $A$ and
  $B$ are Symbols created with \texttt{commutative=False} then SymPy will keep
  $A\cdot B$ and $B\cdot A$ distinct.}. Assumptions on any object can be checked with the
\verb|is_|\texttt{\textit{assumption}} attributes, like \verb|x.is_positive|.

Assumptions are only needed to restrict a domain so that certain
simplifications can be performed. It is not required to make the domain match
the input of a function. For instance, one can create the object
$\sum_{n=0}^m f(n)$ as \verb|Sum(f(n), (n, 0, m))| without setting
\texttt{integer=True} when creating the Symbol object \texttt{n}.

The assumptions system additionally has deductive capabilities. The
assumptions use a three-valued logic using the Python builtin objects
\texttt{True}, \texttt{False}, and \texttt{None}. \texttt{None} represents the
``unknown'' case. This could mean that the given assumption could be either
true or false under the given information, for instance,
\verb|Symbol('x', real=True).is_positive| will give \texttt{None} because a real
symbol might be positive or it might not. It could also mean not enough is
implemented to compute the given fact, for instance,
\verb|(pi + E).is_irrational| gives \texttt{None}, because SymPy does not know
how to determine if $\pi + e$ is rational or irrational, indeed, it is an open
problem in mathematics. % ref?


Basic implications between the facts are used to deduce assumptions. For
instance, the assumptions system knows that being an integer implies being
rational, so \verb|Symbol('x', integer=True).is_rational| returns
\texttt{True}. Furthermore, expressions compute the assumptions on themselves
based on the assumptions of their arguments. For instance, if \texttt{x} and
\texttt{y} are both created with \texttt{positive=True}, then \verb|(x + y).is_positive|
will be \texttt{True}.

SymPy also has an experimental assumptions system where facts are stored
separate from objects, and deductions are made with a SAT solver. We will not
discuss this system here.

%%
%% Extensibility
\subsection{Extensibility}

Extensibility is an important feature for SymPy. Because the same language,
Python, is used both for the internal implementation and the external usage by
users, all the extensibility capabilities available to users are also used by
functions that are part of SymPy.

The typical way to create a custom SymPy object is to subclass an existing
SymPy class, generally either \texttt{Basic}, \texttt{Expr}, or
\texttt{Function}. All SymPy classes used for expression trees~\footnote{Some
  internal classes, such as those used in the polynomial module, do not follow
  this rule.} should be subclasses of the base class
\texttt{Basic}, which defines some basic methods for symbolic expression
trees. \texttt{Expr} is the subclass for mathematical expressions that can be
added and multiplied together. Instances of \texttt{Expr} are typically
complex numbers, but may also include other ``rings'' like matrix expressions.
Not all SymPy classes are \texttt{Expr}. For instance, logic expressions, such
as \verb|And(x, y)| are \texttt{Basic} but not \texttt{Expr}.

The \texttt{Function} class is a subclass of \texttt{Expr} which makes it
easier to define mathematical functions called with arguments. This includes
named functions like $\sin(x)$ and $\log(x)$ as well as undefined functions
like $f(x)$. Subclasses of \texttt{Function} should define a
class method \texttt{eval}, which returns values for which the function should
be automatically evaluated, and \texttt{None} for arguments that shouldn't be
automatically evaluated.

The behavior of classes in SymPy with various other SymPy functions is defined
by defining a relevant \verb|_eval_|\texttt{\textit{*}} method on the class. For instance, an
object can tell SymPy's \texttt{diff} function how to take the derivative of
itself by defining the \verb|_eval_derivative(self, x)| method. The most
common \verb|_eval_|\texttt{\textit{*}} methods relate to the assumptions.
\verb|_eval_is_|\texttt{\textit{assumption}} defines the assumptions for
\textit{assumption}.

Here is a stripped down version of the gamma function $\Gamma(x)$ from SymPy,
which evaluates itself on positive integer arguments, has the positive and
real assumptions defined, can be rewritten in terms of factorial with
\verb|gamma(x).rewrite(factorial)|, and can be differentiated.
\texttt{fdiff} is a convenience method for subclasses of \texttt{Function}.
\texttt{fdiff} returns the derivative of the function without worrying about
the chain rule. \texttt{self.func} is used throughout instead of referencing
\texttt{gamma} explicitly so that potential subclasses of \texttt{gamma} can
reuse the methods.

\begin{verbatim}
from sympy import Integer, Function, floor, factorial, polygamma

class gamma(Function)
    @classmethod
    def eval(cls, arg):
        if isinstance(arg, Integer) and arg.is_positive:
            return factorial(arg - 1)

    def _eval_is_real(self):
        x = self.args[0]
        # noninteger means real and not integer
        if x.is_positive or x.is_noninteger:
            return True

    def _eval_is_positive(self):
        x = self.args[0]
        if x.is_positive:
            return True
        elif x.is_noninteger:
            return floor(x).is_even

    def _eval_rewrite_as_factorial(self, z):
        return factorial(z - 1)

    def fdiff(self, argindex=1):
        from sympy.core.function import ArgumentIndexError
        if argindex == 1:
            return self.func(self.args[0])*polygamma(0, self.args[0])
        else:
            raise ArgumentIndexError(self, argindex)
\end{verbatim}

The actual gamma function defined in SymPy has much more implemented than
this, such as evaluation at rational points and series expansion.


\section{Algorithms}

%% Description of some algorithms (example: integration with Risch, Meijer G, Gruntz, polys)
%%
%% Description of numerics/mpmath (Fredrik)

% A description of some of the algorithms in SymPy. The list is not
% exhaustive.

% The sections here are preliminary. We may end up needing to cut some of
% this.

% XXX: Perhaps this should just be integrated into the features section.

\subsection{Numerics}

The \texttt{Float} class holds an arbitrary-precision binary floating-point value
and a precision in bits. An operation between two \texttt{Float}
inputs is rounded to the larger of the two precisions.
Since Python floating-point literals automatically evaluate to \texttt{double}
(53-bit) precision, strings should be used to input precise decimal values:

\begin{verbatim}
>>> Float(1.1)
1.10000000000000
>>> Float(1.1, 30)   # precision equivalent to 30 digits
1.10000000000000008881784197001
>>> Float("1.1", 30)
1.10000000000000000000000000000
\end{verbatim}

The preferred way to evaluate an expression numerically is with the
\texttt{evalf} method, which internally estimates the number of accurate
bits of the floating-point
approximation for each sub-expression, and adaptively increases the
working precision until the estimated accuracy of the
final result matches the sought number of decimal digits.

The internal error tracking does not provide rigorous error bounds
(in the sense of interval arithmetic) and cannot be used to track
uncertainty in measurement data in any meaningful way;
the sole purpose is to mitigate loss of accuracy that typically occurs
when converting symbolic expressions to numerical values, for example
due to catastrophic cancellation. This is illustrated by the following
example (the input 25 specifies that 25 digits are sought):

\begin{verbatim}
>>> cos(exp(-100)).evalf(25) - 1
0
>>> (cos(exp(-100)) - 1).evalf(25)
-6.919482633683687653243407e-88
\end{verbatim}

The \texttt{evalf} method works with complex numbers and supports
more complicated expressions, such as
special functions, infinite series and integrals.

SymPy does not track the accuracy of
approximate numbers outside of \texttt{evalf}.
The familiar dangers of floating-point arithmetic apply~\cite{goldberg1991every}, and
symbolic expressions containing floating-point numbers should be treated
with some caution.
This approach is similar to Maple and Maxima.

By contrast, Mathematica uses a form
of significance arithmetic~\cite{Sofroniou2005precise} for approximate numbers.
This offers further protection against numerical errors,
but leads to non-obvious semantics while
still not being mathematically rigorous
(for a critique of significance arithmetic, see Fateman~\cite{Fateman1992}).
SymPy's \texttt{evalf} internals are non-rigorous in the same sense,
but have no bearing on the semantics of floating-point
numbers in the rest of the system.

\subsubsection{Code generation}

SymPy's \texttt{lambdify} can be used to convert a symbolic expression to a
callable Python function for faster numerical evaluation.
Various back ends are supported. The following example
demonstrates creating a NumPy-based function from
a SymPy expression, which automatically supports
vectorized array evaluation~\cite{van2011numpy}:

\begin{verbatim}
>>> f = lambdify((x, y), sin(x*y)**2, modules='numpy')
>>> from numpy import array
>>> f(array([1,2,3]), array([4,5,6]))
array([ 0.57275002,  0.29595897,  0.56398184])
\end{verbatim}

SymPy can also generate C, C++, Fortran77, Fortran90 and
Octave/Matlab source code, via the \texttt{codegen} function. [document this?]

\subsubsection{The mpmath library}

The implementation of arbitrary-precision floating-point arithmetic
is supplied by the mpmath library, which originally was developed
as a SymPy module but subsequently has been
moved to a standalone Python package. The basic datatypes in mpmath
are \texttt{mpf} and \texttt{mpc}, which respectively act as
multiprecision substitutes for Python's \texttt{float} and \texttt{complex}.
The floating-point precision is controlled by a global context:

\begin{verbatim}
>>> import mpmath
>>> mpmath.mp.dps = 30    # 30 digits of precision
>>> mpmath.mpf("0.1") + mpmath.exp(-50)
mpf('0.100000000000000000000192874984794')
>>> print(_)   # pretty-printed
0.100000000000000000000192874985
\end{verbatim}

For pure numerical computing, it is convenient to use mpmath directly
with \texttt{from mpmath import *} (it is best to avoid such an
import statement when using SymPy simultaneously, since numerical
functions such as \texttt{exp} will shadow the symbolic counterparts
in SymPy).

Like SymPy, mpmath is a pure Python library.
Internally, mpmath represents a floating-point number
${(-1)}^s x \cdot 2^y$ by a tuple $(s, x, y, b)$ where
$x$ and $y$ are arbitrary-size Python integers
and the redundant integer $b$ stores the bit length of $x$ for quick access.
If GMPY~\cite{GMPY} is installed, mpmath automatically switches to
using the \texttt{gmpy.mpz} type for $x$ and using GMPY helper methods
to perform rounding-related operations, improving performance.

The mpmath library includes support for
special functions, root-finding, linear algebra, polynomial approximation,
and numerical computation of limits, derivatives, integrals, infinite
series, and ODE solutions. All features work in arbitrary precision
and use algorithms that support computing hundreds of digits rapidly,
except in degenerate cases.

The double exponential (tanh-sinh) quadrature is used for numerical
integration by default. For smooth integrands, this algorithm usually
converges extremely rapidly, even when the integration interval is infinite
or singularities are present at the endpoints~\cite{takahasi1974double,bailey2005comparison}.
However, for good performance, singularities
in the middle of the interval must be specified
by the user.
To evaluate slowly converging limits and infinite series, mpmath
automatically attempts to apply Richardson extrapolation and the
Shanks transformation
(Euler-Maclaurin summation can also be used)~\cite{BenderOrszag1999}.
A function to evaluate oscillatory integrals by means of convergence
acceleration is also available.

A wide array of higher mathematical functions are implemented
with full support for complex values of all parameters and arguments,
including complete and incomplete gamma functions,
Bessel functions, orthogonal polynomials, elliptic functions and integrals,
zeta and polylogarithm functions,
the generalized hypergeometric function, and the Meijer G-function.

Most special functions are implemented as linear
combinations of the generalized hypergeometric function ${}_{p}F_{q}$,
which is computed by a combination of direct summation,
argument transformations (for ${}_2F_1$, ${}_3F_2$, $\ldots$)
and asymptotic expansions
(for ${}_0F_1$, ${}_1F_1$, ${}_1F_2$, ${}_2F_2$, ${}_2F_3$)
to cover the whole complex domain.
Numerical integration and generic convergence acceleration
are also used in a few special cases.

In general, linear combinations and argument transformations
give rise to singularities that have to be removed for certain
combinations of parameters.
A typical example is the modified Bessel function of the second kind
$$K_{\nu}(z) = \frac{1}{2} \left[
            {\left(\frac{z}{2}\right)}^{-\nu}
                \Gamma(\nu)
                {}_0F_1\!\left(1-\nu, \frac{z^2}{4}\right)
             -
             {\left(\frac{z}{2}\right)}^{\nu}
                 \frac{\pi}{\nu \sin(\pi \nu) \Gamma(\nu)}
                 {}_0F_1\!\left(\nu+1, \frac{z^2}{4}\right)
            \right]$$
where the limiting value $\lim_{\varepsilon \to 0} K_{n+\varepsilon}(z)$
has to be computed when $\nu = n$ is an integer.
A generic algorithm is used to evaluate
hypergeometric-type linear combinations of the above type.
This algorithm automatically detects cancellation problems,
and computes limits numerically by perturbing parameters whenever
internal singularities occur (the perturbation size is automatically
decreased until the result is detected to converge numerically).

Due to this generic approach, particular combinations of hypergeometric
functions can be specified easily.
The implementation of the Meijer G-function takes only a few dozen lines of
code, yet covers the whole input domain in a robust way.
The Meijer G-function instance
$G_{1, 3}^{3, 0}\left(0 ; \tfrac{1}{2}, -1, - \tfrac{3}{2} | x \right)$
is a good test case~\cite{Toth2007}; past versions of both Maple and
Mathematica produced incorrect numerical values for large $x > 0$.
Here, mpmath automatically removes the internal singularity
and compensates for cancellations (amounting to 656 bits
of precision when $x = 10000$), giving correct values:
\begin{verbatim}
>>> mpmath.mp.dps = 15
>>> mpmath.meijerg([[],[0]],[[-0.5,-1,-1.5],[]],10000)
mpf('2.4392576907199564e-94')
\end{verbatim}

Equivalently, with SymPy's interface this function can be evaluated as:
\begin{verbatim}
>>> meijerg([[],[0]],[[-S(1)/2,-1,-S(3)/2],[]],10000).evalf()
2.43925769071996e-94
\end{verbatim}

We highlight the generalized hypergeometric functions and
the Meijer G-function, due to those functions' frequent appearance
in closed forms for integrals and sums [todo: crossref symbolic integration].
Via mpmath, SymPy has relatively good support for evaluating sums and integrals
numerically, using two complementary approaches: direct numerical evaluation,
or first computing a symbolic closed form involving special functions. [example?]

\subsubsection{Numerical simplification}

The \texttt{nsimplify} function in SymPy
(a wrapper of \texttt{identify} in mpmath)
attempts to find a simple symbolic
expression that evaluates to the same numerical value as the given
input.
It works by applying a few simple transformations
(including square roots, reciprocals, logarithms and exponentials) to
the input and, for each transformed value,
using the PSLQ algorithm~\cite{Ferguson1999} to search for
a matching algebraic number or optionally a linear combination
of user-provided base constants (such as $\pi$).

\begin{verbatim}
>>> x = 1 / (sin(pi/5)+sin(2*pi/5)+sin(3*pi/5)+sin(4*pi/5))**2
>>> nsimplify(x)
-2*sqrt(5)/5 + 1
>>> nsimplify(pi, tolerance=0.01)
22/7
>>> nsimplify(1.783919626661888, [pi], tolerance=1e-12)
pi/(-1/3 + 2*pi/3)
\end{verbatim}

\subsection{Polynomials}

\subsection{The Risch Algorithm}
% Also the Meijer-G algorithm, if someone can write about it

\subsection{The Gruntz Algorithm}

The limit module implements the Gruntz algorithm
\cite{Gruntz1996limits}.

Examples:
\begin{verbatim}
In [1]: limit(sin(x)/x, x, 0)
Out[1]: 1

In [2]: limit((2*E**((1-cos(x))/sin(x))-1)**(sinh(x)/atan(x)**2), x, 0)
Out[2]: E
\end{verbatim}

\subsubsection{Details}

We first define comparability classes by calculating $L$:
\begin{equation}
L\equiv \lim_{x\to\infty} {\log |f(x)| \over \log |g(x)|}
\end{equation}
And then we define the $<$, $>$ and $\sim$ operations as follows: $f>g$ when
$L=\pm\infty$ ($f$ is more rapidly varying than $g$, i.e., $f$ goes to $\infty$
or $0$ faster than $g$, $f$ is greater than any power of $g$), $f<g$ when $L=0$
($f$ is less rapidly varying than $g$) and $f\sim g$ when $L\neq 0,\pm\infty$
(both $f$ and $g$ are bounded from above and below by suitable integral powers
of the other).

Examples:
$$2 < x < e^x < e^{x^2} < e^{e^x}$$
$$2\sim 3\sim -5$$
$$x\sim x^2\sim x^3\sim {1\over x}\sim x^m\sim -x$$
$$e^x\sim e^{-x}\sim e^{2x}\sim e^{x+e^{-x}}$$
$$f(x)\sim{1\over f(x)}$$

The Gruntz algorithm, on an example:
$$f(x) = e^{x+2e^{-x}} - e^x + {1\over x}$$
$$\lim_{x\to\infty} f(x) = ?$$

Strategy:
mrv set: the set of most rapidly varying subexpressions
$\{e^x, e^{-x}, e^{x+2e^{-x}}\}$, the same comparability class
Take an item $\omega$ from mrv, converging to 0 at infinity. Here
$\omega=e^{-x}$. If not present in the mrv set, use the relation
$f(x)\sim {1\over f(x)}$.

Rewrite the mrv set using $\omega$: $\{{1\over\omega}, \omega,
{1\over\omega}e^{2\omega}\}$, substitute back into $f(x)$ and expand in
$\omega$:
$$f(x) = {1\over x}-{1\over\omega}+{1\over\omega}e^{2\omega}
    = 2+{1\over x} + 2\omega + O(\omega^2)$$

The core idea of the algorithm: $\omega$ is from the mrv set, so in the limit
$\omega\to0$:
$$f(x) = {1\over x}-{1\over\omega}+{1\over\omega}e^{2\omega}
    = 2+{1\over x} + 2\omega + O(\omega^2)
    \to 2 + {1\over x}$$

We iterate until we get just a number, the final limit. Gruntz proved this
algorithm always works and converges in his Ph.D. thesis
\cite{Gruntz1996limits}.

Generally:
$$ f(x) = \underbrace{O\left({1\over \omega^3}\right)}_\infty
    + \underbrace{C_{-2}(x)\over \omega^2}_\infty
    + \underbrace{C_{-1}(x)\over \omega}_\infty
    + {C_{0}(x)}
    + \underbrace{C_{1}(x)\omega}_0
    + \underbrace{O(\omega^2)}_0
$$
we look at the lowest power of $\omega$. The limit is one of: $0$,
$\lim_{x\to\infty} C_0(x)$, $\infty$.

\subsection{Logic}

\subsection{Other}


\section{Features}

%% List of Features and how to use
%%
%% Quick overview of the main modules, what it can do and so on. It should probably provide examples how to use sympy.
%%
%% See also the supplement (below)

% Features to discuss in-depth:

SymPy has an extensive feature set that encompasses too much to cover
in-depth here. Bedrock areas, such a Calculus, receive their own sub-sections
below. Additionally, Table~\ref{features-table} describes other capabilities
present in the SymPy code base. This gives a sampling from the breadth of
topics and application domains that SymPy services.


\begin{longtable}[htbc]{|l|p{0.7\linewidth}|}
\caption{SymPy Features and Descriptions\label{features-table}}\\
\hline
\textbf{Feature} & \textbf{Description} \\
\hline
Discrete Math & Summations, products, binomial coefficients,
    prime number tools, integer factorization, Diophantine equation solving, and
    boolean logic representation, equivalence testing, and inference.\\
Concrete Math & Tools for determining whether summation and product
    expressions are convergent, absolutely convergent, hypergeometric, and
    other properties. May also compute Gosper's normal form~\cite{petkovvsek1996bak} for two univariate polynomials.\\
Plotting & Hooks for visualizing expressions via matplotlib~\cite{Hunter:2007}
    or as text drawings when lacking a graphical back-end.\\
Geometry & Allows the creation of 2D geometrical entities,
    such as lines and circles. Enables queries on these entities, including
    asking the area of an ellipse, checking for collinearity of a set of
    points, or finding the intersection between two lines.\\
Statistics & Support for a random variable type as well as the ability to
    declare this variable from prebuilt distribution functions such as
    Normal, Exponential, Coin, Die, and other custom distributions.\\
Polynomials & Computes polynomial algebras over various coefficient domains
    ranging from the simple (e.g., polynomial division) to the advanced
    (e.g., Gr\"obner bases~\cite{adams1994introduction} and multivariate
    factorization over algebraic number domains).\\
Sets & Representations of empty, finite, and infinite sets. This includes
    special sets such as for all natural, integer, and complex numbers.\\
Series & Implements series expansion, sequences, and limit of sequences.
    This includes special series, such as Fourier and power series.\\
Vectors & Provides basic vector math and differential calculus with respect
    to 3D Cartesian coordinate systems.\\
Matrices & Tools for creating matrices of symbols and expressions.
    This is capable of both sparse and dense representations and performing
    symbolic linear algebraic operations (e.g., inversion and factorization).\\
Combinatorics \& Group Theory & Implements permutations, combinations,
    partitions, subsets,
    various permutation groups (such as polyhedral, Rubik, symmetric,
    and others), Gray codes~\cite{Nijenhuis1978combinatorial},
    and Prufer sequences~\cite{biggs1976graph}.\\
Code Generation & Enables generation of compilable and executable
    code in a variety of different programming languages directly from
    expressions. Target languages include C, Fortran, Julia, JavaScript,
    Mathematica, Matlab and Octave, Python, and Theano.\\
Tensors & Symbolic manipulation of indexed objects.\\
Lie Algebras & Represents Lie algebras and root systems.\\
Cryptography & Represents block and stream ciphers, including
    shift, Affine, substitution, Vigenere's, Hill's, bifid, RSA, Kid RSA,
    linear-feedback shift registers, and Elgamal encryption\\
Special Functions & Implements a number of well known special functions,
    including Dirac delta, Gamma, Beta, Gauss error functions, Fresnel
    integrals, Exponential integrals, Logarithmic integrals, Trigonometric
    integrals, Bessel, Hankel, Airy, B-spline, Riemann Zeta, Dirichlet eta,
    polylogarithm, Lerch transcendent, hypergeometric, elliptic integrals,
    Mathieu, Jacobi polynomials, Gegenbauer polynomial, Chebyshev polynomial,
    Legendre polynomial, Hermite polynomial, Laguerre polynomial, and
    spherical harmonic functions.\\
\hline

\end{longtable}


% Basic operations (the core)
\subsection{Basic Operations}

\input{features_basic_operations}

\subsection{Calculus}

% Sets
\subsection{Sets}
%% Sets

SymPy supports representation of a wide variety of sets, this is achieved by
first defining abstract representation for a smaller number of atomic set
classes and then combining and transforming them using various set operations.

Each of the set classes inherits from the base set class and defines rules to
check membership of a SymPy object in that set, to calculate union,
intersection and set difference. In cases we are not able to evaluate these
operations to atomic set classes they are represented as abstract unevaluated
objects.


We have the following atomic set classes in SymPy.

\begin{itemize}

    \item \verb|EmptySet|: represents the empty set $\emptyset$.

    \item \verb|UniversalSet|: Everything is a member of Universal Set.
        Union of Universal Set with any set gives Universal Set and
        intersection leads to the other set itself.

    \item \verb|FiniteSet| is functionally equivalent to python's set
        object. Its members can be any SymPy object including other sets
        themselves.

    \item \verb|Integers| represents set of Integers $\mathbb{Z}$.

    \item \verb|Naturals| represents set of Natural numbers $\mathbb{N}$ i.e.,
        set of positive integers.

    \item \verb|Naturals0| represents the whole numbers which are all the
        non-negative integers, inclusive of zero.

    \item \verb|Range| represents a range of integers and is defined by
        specifying a start value, an end value and a step size. Range is
        functionally equivalent to python's range except the fact that it
        accepts infinity at end points allowing us to represent infinite
        ranges.


    \item \verb|RealInterval| is specified by giving the start and end point
        and specifying if it is open or closed in the respective ends. The set
        of real numbers is represented as a special case of a real interval
        where the start point is negative infinite and the end point is
        positive infinite.


\end{itemize}


%% Operations

Other than unevaluated classes of Union, Intersection and Set Difference
operations, we have following set classes.

\begin{itemize}

    \item \verb|ProductSet| abstractly defines the Cartesian product of two
        or more sets. Product Set is useful when representing higher
        dimensional spaces. For example to represent a three dimensional space
        we simply take the Cartesian product of three Real sets.

    \item \verb|ImageSet| represents the image of a function when applied to a
        particular set. In notation Image Set of a function $F$ w.r.t a set $S$
        is $\{ F(x) | x \in S \}$ In particular we use Image Set to represent
        the set of infinite solutions from trigonometric equations.


    \item \verb|ConditionSet| represents subset of a set who's members
        satisfies a particular condition. In notation Condition Set of set $S$
        w.r.t to a condition $H$ is $\{x | H(x), x \in S \}$. We use Condition Set
        to represent the set of solutions of an equation or an inequality where
        the equation or the inequality is the condition and the set is the
        domain in which we aim to find the solution.


\end{itemize}

A few other classes are implemented as special cases of the classes described
above. The real number \verb|Reals| is implemented as a special case of real
interval where the start point is negative infinity and the end point is
positive infinity. \verb|ComplexRegion| is implemented as a special case of
\verb|ImageSet|, \verb|ComplexRegion| supports both polar and rectangular
representation of region on the complex plane.


% Solvers (regular equations, maybe also mention other types of solvers like ODEs/recurrence/Diophantine)
\subsection{Solvers}
%% Solvers in SymPy


SymPy has module of equation solvers for symbolic equations. There are two
submodules to solve algebraic equations in SymPy, referred to as old solve
function, \texttt{solve}, and new solve function, \texttt{solveset}.
Solveset is introduced with several design changes with respect to old
\texttt{solve} function to resolve the issues with old \texttt{solve} function,
for example old \texttt{solve} function's input API has many flags which are
not needed and they make it hard for the user and the developers to work on
solvers. In contrast to old solve function, the \texttt{solveset} has a clean
input API, It only asks for the much needed information from the user, following
are the function signatures of old and new solve function:

\begin{verbatim}
solve(f, *symbols, **flags)  # old solve function
solveset(f, symbol, domain)  # new solve function
\end{verbatim}

The old \texttt{solve} function has an inconsistent output API for various types
of inputs, whereas the \texttt{solveset} has a canonical output API which is
achieved using sets. It can consistently return various types of solutions.

\begin{itemize}
\item Single solution
\end{itemize}
\begin{verbatim}
>>> solveset(x - 1)
>>> {1}
\end{verbatim}

\begin{itemize}
\item Finite set of solution, quadratic equation
\end{itemize}
\begin{verbatim}
>>> solveset(x**2 - pi**2, x)
{-pi, pi}
\end{verbatim}

\begin{itemize}
\item No Solution
\end{itemize}
\begin{verbatim}
>>> solveset(1, x)
EmptySet()
\end{verbatim}

\begin{itemize}
\item Interval of solution
\end{itemize}
\begin{verbatim}
>>> solveset(x**2 - 3 > 0, x, domain=S.Reals)
(-oo, -sqrt(3)) U (sqrt(3), oo)
\end{verbatim}

\begin{itemize}
\item Infinitely many solutions
\end{itemize}
\begin{verbatim}
>>> solveset(sin(x) - 1, x, domain=S.Reals)
ImageSet(Lambda(_n, 2*_n*pi + pi/2), Integers())
>>> solveset(x - x, x, domain=S.Reals)
(-oo, oo)
>>> solveset(x - x, x, domain=S.Complexes)
S.Complexes
\end{verbatim}

\begin{itemize}
\item Linear system: finite and infinite solution for determined, under
determined and over determined problems.
\end{itemize}
\begin{verbatim}
>>> A = Matrix([[1, 2, 3], [4, 5, 6], [7, 8, 10]])
>>> b = Matrix([3, 6, 9])
>>> linsolve((A, b), x, y, z)
{(−1,2,0)}
>>> linsolve(Matrix(([1, 1, 1, 1], [1, 1, 2, 3])), (x, y, z))
{(-y - 1, y, 2)}
\end{verbatim}

The new solve i.e. \textbf{solveset} is under active development and is a
planned replacement for \textbf{solve}, Hence there are some features which are
implemented in solve and is not yet implemented in solveset. The table below
show the current state of old and new solve functions.

\hfill

\begin{tabular}{ |p{4cm}|p{3cm}|p{3cm}|  }
\hline
\multicolumn{3}{|c|}{Solveset vs Solve} \\
\hline
Feature& solve &solveset \\
\hline
Consistent Output API & No & Yes \\
Consistent Input API & No & Yes \\
Univariate & Yes & Yes\\
Linear System& Yes & Yes (linsolve) \\
Non Linear System& Yes & Not yet \\
Transcendental& Yes & Not yet \\
\hline
\end{tabular}

\hfill \break{}

Below are some of the examples of old \textbf{solve} function:

\begin{itemize}
\item Non Linear (multivariate) System of Equation: Intersection of a circle
and a parabola.
\end{itemize}
\begin{verbatim}
>>> solve([x**2 + y**2 - 16, 4*x - y**2 + 6], x, y)
[(-2 + sqrt(14), -sqrt(-2 + 4*sqrt(14))),
 (-2 + sqrt(14), sqrt(-2 + 4*sqrt(14))),
 (-sqrt(14) - 2, -I*sqrt(2 + 4*sqrt(14))),
 (-sqrt(14) - 2, I*sqrt(2 + 4*sqrt(14)))]
\end{verbatim}

\begin{itemize}
\item Transcendental Equation
\end{itemize}
\begin{verbatim}
>>> solve(x + log(x))**2 - 5*(x + log(x)) + 6, x)
[LambertW(exp(2)), LambertW(exp(3))]
>>> solve(x**3 + exp(x))
[-3*LambertW((-1)**(2/3)/3)]
\end{verbatim}


Diophantine equations play a central and an important role in number theory.
A Diophantine equation has the form, $f(x_1, x_2, \ldots x_n) = 0$
where $n \geq 2$ and $x_1, x_2, \ldots x_n$ are integer variables. If we can find
$n$ integers $a_1, a_2, \ldots a_n$ such that $x_1 = a_1, x_2 = a_2, \ldots x_n = a_n$
satisfies the above equation, we say that the equation is solvable.

Currently, following five types of Diophantine equations can be solved using
SymPy's Diophantine module.

\begin{itemize}
    \item Linear Diophantine equations: $a_1x_1 + a_2x_2 + \cdots + a_{n}x_{n} = b$
    \item General binary quadratic equation: $ax^2 + bxy + cy^2 + dx + ey + f = 0$
    \item Homogeneous ternary quadratic equation: $ax^2 + by^2 + cz^2 + dxy + eyz + fzx = 0$
    \item Extended Pythagorean equation: $a_{1}x_{1}^2 + a_{2}x_{2}^2 + \cdots + a_{n}x_{n}^2 = a_{n+1}x_{n+1}^2$
    \item General sum of squares: $x_{1}^2 + x_{2}^2 + \cdots + x_{n}^2 = k$
\end{itemize}

When an equation is fed into Diophantine module, it factors the equation (if
possible) and solves each factor separately. Then all the results are combined to
create the final solution set. Following examples illustrate some of the basic
functionalities of the Diophantine module.

\begin{verbatim}
>>> from sympy import symbols
>>> x, y, z = symbols("x, y, z", integer=True)

>>> diophantine(2*x + 3*y - 5)
set([(3*t_0 - 5, -2*t_0 + 5)])

>>> diophantine(2*x + 4*y - 3)
set()

>>> diophantine(x**2 - 4*x*y + 8*y**2 - 3*x + 7*y - 5)
set([(2, 1), (5, 1)])

>>> diophantine(x**2 - 4*x*y + 4*y**2 - 3*x + 7*y - 5)
set([(-2*t**2 - 7*t + 10, -t**2 - 3*t + 5)])

>>> diophantine(3*x**2 + 4*y**2 - 5*z**2 + 4*x*y - 7*y*z + 7*z*x)
set([(-16*p**2 + 28*p*q + 20*q**2, 3*p**2 + 38*p*q - 25*q**2, 4*p**2 - 24*p*q + 68*q**2)])

>>> from sympy.abc import a, b, c, d, e, f
>>> diophantine(9*a**2 + 16*b**2 + c**2 + 49*d**2 + 4*e**2 - 25*f**2)
set([(70*t1**2 + 70*t2**2 + 70*t3**2 + 70*t4**2 - 70*t5**2, 105*t1*t5, 420*t2*t5, 60*t3*t5, 210*t4*t5, 42*t1**2 + 42*t2**2 + 42*t3**2 + 42*t4**2 + 42*t5**2)])

>>> diophantine(a**2 + b**2 + c**2 + d**2 + e**2 + f**2 - 112)
set([(8, 4, 4, 4, 0, 0)])
\end{verbatim}


% Matrices (worth including to stress that they are symbolic)
\subsection{Matrices}

\input{features_matrices}

% Physics module (some sampling, to show that it is there)
\subsection{Physics}

% Logic module
\subsection{Logic}

SymPy supports construction and manipulation of boolean expressions
through the \texttt{logic} module. SymPy symbols can be used as
propositional variables and also be substituted as \texttt{True}
or \texttt{False}. A good number of manipulation features for boolean
expressions have been implemented in the \texttt{logic} module.

\subsubsection{Constructing boolean expressions}

A boolean variable can be declared as a SymPy symbol. Python
operators \&, \textbar{} and \textasciitilde{} are overloaded for logical \texttt{And},
\texttt{Or} and \texttt{negate}. Several others like \texttt{Xor},
\texttt{Implies} can be constructed with \^{}, \textgreater\textgreater{} respectively.
The above are just a shorthand, expressions can also be constructed
by directly calling \verb|And()|, \verb|Or()|, \verb|Not()|,
\verb|Xor()|, \verb|Nand()|, \verb|Nor()|, etc.

\begin{verbatim}
>>> from sympy import *
>>> x, y, z = symbols('x y z')
>>> e = (x & y) | z
>>> e.subs({x: True, y: True, z: False})
True
\end{verbatim}

\subsubsection{CNF and DNF}

Any boolean expression can be converted to conjunctive normal
form, disjunctive normal form and negation normal form. The
API also permits to check if a boolean expression is in any
of the above mentioned forms.

\begin{verbatim}
>>> from sympy import *
>>> x, y, z = symbols('x y z')
>>> to_cnf((x & y) | z)
And(Or(x, z), Or(y, z))
>>> to_dnf(x & (y | z))
Or(And(x, y), And(x, z))
>>> is_cnf((x | y) & z)
True
>>> is_dnf((x & y) | z)
True
\end{verbatim}

\subsubsection{Simplification and Equivalence}

The module supports simplification of given boolean expression
by making deductions on it. Equivalence of two expressions can
also be checked. If so, it is possible to return the mapping of
variables of two expressions so as to represent the
same logical behaviour.

\begin{verbatim}
>>> from sympy import *
>>> a, b, c, x, y, z = symbols('a b c x y z')
>>> e = a & (~a | ~b) & (a | c)
>>> simplify(e)
And(Not(b), a)
>>> e1 = a & (b | c)
>>> e2 = (x & y) | (x & z)
>>> bool_map(e1, e2)
(And(Or(b, c), a), {b: y, a: x, c: z})
\end{verbatim}

\subsubsection{SAT solving}

The module also supports satisfiability checking of a given
boolean expression. If satisfiable, it is possible to return
a model for which the expression is satisfiable. The API also
supports returning all possible models. The SAT solver has
a clause learning DPLL algorithm implemented with watch
literal scheme and VSIDS heuristic\cite{moskewicz2008method}.

\begin{verbatim}
>>> from sympy import *
>>> a, b, c = symbols('a b c')
>>> satisfiable(a & (~a | b) & (~b | c) & ~c)
False
>>> satisfiable(a & (~a | b) & (~b | c) & c)
{b: True, a: True, c: True}
\end{verbatim}



SymPy includes several packages that allow users to solve domain specific
problems. For example, a comprehensive physics package is included that is
useful for solving problems in classical mechanics, optics, and quantum
mechanics along with support for manipuating physical quantities with units.

\subsection{Vector Algebra}

The \verb|sympy.physics.vector| package provides reference frame, time, and
space aware vector and dyadic objects that allow for three dimensional
operations such as addition, subtraction, scalar multiplication, inner and
outer products, cross products, etc. Both of these objects can be written in
very compact notation that make it easy to express the vectors and dyadics in
terms of multiple reference frames with arbitrarily defined relative
orientations. The vectors are used to specify the positions, velocities, and
accelerations of points, orientations, angular velocities, and angular
accelerations of reference frames, and force and torques. The dyadics are
essentially reference frame aware $3 \times 3$ tensors. The vector and dyadic
objects can be used for any one-, two-, or three-dimensional vector algebra and
they provide a strong framework for building physics and engineering tools.

\begin{listing}
  \begin{minted}{pycon}
>>> from sympy import pi
>>> from sympy.physics.vector import ReferenceFrame
>>> A = ReferenceFrame('A')
>>> B = ReferenceFrame('B')
>>> C = ReferenceFrame('C')
>>> B.orient(A, 'body', (pi, pi / 3, pi / 4), 'zxz')
>>> C.orient(B, 'axis', (pi / 2, B.x))
>>> v = 1 * A.x + 2 * B.z + 3 * C.y
>>> v
A.x + 2*B.z + 3*C.y
>>> v.express(A)
A.x + 5*sqrt(3)/2*A.y + 5/2*A.z
  \end{minted}
  \caption{
    Python interpreter session showing how a vector is created using the
    orthogonal unit vectors of three reference frames that are oriented with
    respect to each other and the result of expressing the vector in the $A$
    frame.
    The $B$ frame is oriented with respect to the $A$ frame using Z-X-Z Euler
    Angles of magnitude $\pi$, $\frac{\pi}{2}$, and
    $\frac{\pi}{3}$\si{\radian},
    respectively whereas the $C$ frame is oriented with respect to the $B$
    frame through a simple rotation about the $B$ frame's X unit vector through
    $\frac{\pi}{2}$\si{\radian}.}
  \label{lis:physics-vector}
\end{listing}

\subsection{Classical Mechanics}

The \verb|physics.mechanics| package utilizes the \verb|physics.vector| package
to populate time aware particle and rigid body objects to fully describe the
kinematics and kinetics of a rigid multi-body system. These objects store all
of the information needed to derive the ordinary differential or differential
algebraic equations that govern the motion of the system, i.e., the equations
of motion. These equations of motion abide by Newton's laws of motion and can
handle any arbitrary kinematical constraints or complex loads. The package
offers two automated methods for formulating the equations of motion based on
Lagrangian Dynamics~\cite{Lagrange1811} and Kane's Method~\cite{Kane1985}. Lastly, there
are automated linearization routines for constrained dynamical
systems based on~\cite{Peterson2014}.

\subsection{Quantum Mechanics}

The \verb|sympy.physics.quantum| package provides quantum functions, states,
operators, and computation of standard quantum models.

% TODO : This needs some help from someone that knows something about quantum
% physics. I wasn't able to understand much from the documentation.

\subsection{Optics}

The \verb|physics.optics| package provides Gaussian optics functions.

% TODO : This needs some help from someone that knows something about optics.

\subsection{Units}

The \verb|physics.units| module provides around two hundred predefined prefixes
and SI units that are commonly used in the sciences. Additionally, it provides
the \verb|Unit| class which allows the user to define their own units.  These
prefixes and units are multiplied by standard SymPy objects to make expressions
unit aware, allowing for algebraic and calculus manipulations to be applied to
the expressions while the units are tracked in the manipulations.  The units of
the expressions can be easily converted to other desired units.  There is also
a new units system in \verb|sympy.physics.unitsystems| that allows the user to
work in specified unit systems.


\section{Other Projects that use SymPy}

There are several projects that use SymPy as a library for implementing
a part of their project, or even as a part of back-end for their
application as well.
\newline
Some of them are listed below:

\begin{itemize}
\item
  \href{http://cadabra.science/index.html}{\textbf{Cadabra}}: Cadabra is
  a symbolic computer algebra system (CAS) designed specifically for the
  solution of problems encountered in field theory.
\item
  \href{http://octave.sourceforge.net/symbolic/}{\textbf{Octave Symbolic}}:
  The Octave-Forge Symbolic package adds symbolic calculation features
  to GNU Octave. These include common Computer Algebra System tools such
  as algebraic operations, calculus, equation solving, Fourier and
  Laplace transforms, variable precision arithmetic and other features.
\item
  \href{https://github.com/jverzani/SymPy.jl}{\textbf{SymPy.jl}}:
  Provides a Julia interface to SymPy using PyCall.
\item
  \href{https://mathics.github.io/}{\textbf{Mathics}}: Mathics is a
  free, general-purpose online CAS featuring Mathematica compatible
  syntax and functions. It is backed by highly extensible Python code,
  relying on SymPy for most mathematical tasks.
\item
  \href{http://mathpix.com/}{\textbf{Mathpix}}: An iOS App, that uses
  Artificial Intelligence to detect handwritten math as input, and uses
  SymPy Gamma, to evaluate the math input and generate the relevant
  steps to solve the problem.
\item
  \href{http://www.sagemath.org/}{\textbf{Sage}}: A CAS, visioned to be
  a viable free open source alternative to Magma, Maple, Mathematica and
  Matlab.
\item
  \href{https://cloud.sagemath.com}{\textbf{SageMathCloud}}:
  SageMathCloud is a web-based cloud computing and course management
  platform for computational mathematics.
\item
  \href{http://www.pydy.org/}{\textbf{PyDy}}: Multibody Dynamics with
  Python.
\item
  \href{https://github.com/brombo/galgebra}{\textbf{galgebra}}:
  Geometric algebra (previously sympy.galgebra).
\item
  \href{http://yt-project.org/}{\textbf{yt}}: Python package for
  analyzing and visualizing volumetric data (yt.units uses SymPy).
\item
  \href{http://sfepy.org/}{\textbf{SfePy}}: Simple finite elements in
  Python.
\item
  \href{http://quameon.sourceforge.net/}{\textbf{Quameon}}: Quantum
  Monte Carlo in Python.
\item
  \href{http://lcapy.elec.canterbury.ac.nz/}{\textbf{Lcapy}}:
  Experimental Python package for teaching linear circuit analysis.
\item
  \href{http://digitalcommons.calpoly.edu/cgi/viewcontent.cgi?article=1072\&context=physsp/}{\textbf{Quantum
  Programming in Python}}: Quantum 1D Simple Harmonic Oscillator and
  Quantum Mapping Gate.
\item
  \href{http://mech.fsv.cvut.cz/~stransky/software/latexexpr/doc/}{\textbf{LaTeX
  Expression project}}: Easy LaTeX typesetting of algebraic expressions
  in symbolic form with automatic substitution and result computation.
\item
  \href{https://www.researchgate.net/publication/260585491_Symbolic_Statistics_with_SymPy/}{\textbf{Symbolic
  statistical modeling}}: Adding statistical operations to complex
  physical models.
\end{itemize}

\subsection{SymPy Gamma}\label{sympy-gamma}

SymPy Gamma is a simple web application that runs on Google App Engine.
It executes and displays the results of SymPy expressions as well as
additional related computations, in a fashion similar to that of
Wolfram\textbar{}Alpha. For instance, entering an integer will display
its prime factors, digits in the base-10 expansion, and a factorization
diagram. Entering a function will display its docstring; in general,
entering an arbitrary expression will display its derivative, integral,
series expansion, plot, and roots.

SymPy Gamma also has several additional features than just computing the
results using SymPy.

\begin{itemize}
\item
  It displays integration steps, differentiation steps in detail, which
  can be viewed in Figure~\ref{fig:integralsteps}:\par
\begin{minipage}{\textwidth}
    \centering
    \includegraphics[width=0.7\textwidth]{integral_steps.png}
    \captionof{figure}{Integral steps of $\tan (x)$}\label{fig:integralsteps}
\end{minipage}
\item
  It also displays the factor tree diagrams for different numbers.
\item
  SymPy Gamma also saves user search queries, and offers many such
  similar features for free, which Wolfram\textbar{}Alpha only offers
  to its paid users.
\end{itemize}
Every input query from the user on SymPy Gamma is first, parsed by its
own parser, which handles several different forms of function names,
which SymPy as a library doesn't support. For instance, SymPy Gamma
supports queries like \texttt{sin\ x}, whereas SymPy doesn't support
this, and supports only \verb|sin(x)|.

This parser converts the input query to the equivalent SymPy readable
code, which is then eventually processed by SymPy and the result is
finally formatted in LaTeX and displayed on the SymPy Gamma web-application.

\subsection{SymPy Live}\label{sympy-live}

SymPy Live is an online Python shell, which runs on Google
App Engine, that executes SymPy code. It is integrated in the SymPy
documentation examples, located at this \href{http://docs.sympy.org/latest/index.html}{link}.

This is accomplished by providing a HTML/JavaScript GUI for entering
source code and visualization of output, and a server part which
evaluates the requested source code. It's an interactive AJAX shell,
that runs SymPy code using Python on the server.
\newline
Certain Features of SymPy Live:

\begin{itemize}
\item
  It supports the exact same syntax as SymPy, hence it can be used
  easily, to test for outputs of various SymPy expressions.
\item
  It can be run as a standalone app or in an existing app as an
  admin-only handler, and can also be used for system administration
  tasks, as an interactive way to try out APIs, or as a debugging aid
  during development.
\item
  It can also be used to plot figures (\href{http://live.sympy.org/?evaluate=from\%20sympy\%20import\%20symbols\%0Afrom\%20sympy.plotting\%20import\%20textplot\%0Ax\%20\%3D\%20symbols(\%27x\%27)\%0Atextplot(x**2\%2C0\%2C5)\%0A\%23--\%0A}{link}),
  and execute all kinds of expressions that SymPy can evaluate.
\item
SymPy Live also formats the output in LaTeX for pretty-printing the
output.
\end{itemize}


\section{Comparison with other CAS}


\subsection{Mathematica}

Wolfram Mathematica is a popular proprietary CAS.\@
It features highly advanced algorithms.
Mathematica has a core implemented in C++~\cite{Wolfram2016}
which interprets its own programming language (know as Wolfram language).

% M-expressions

Analogously to Lisp's S-expressions,
Mathematica uses its own style of M-expressions,
which are arrays of either atoms or other M-expression.
The first element of the expression identifies the type of the expression
and is indexed by zero, whereas the first argument is indexed by one.
Notice that SymPy expression arguments are stored in a Python tuple
(that is, an immutable array),
while the expression type is identified by the type of the object storing the
expression.

% Attributes

Mathematica can associate attributes to its atoms.

% Expression mutability

Unlike SymPy, Mathematica's expressions are mutable,
that is one can change parts of the expression tree without the need of
creating a new object.
The reactivity of Mathematica allows for a lazy updating of any references
to that data structure.

% Products and commutativity

Products in Mathematica are determined by some builtin node types,
such as \texttt{Times}, \texttt{Dot}, and others.
\texttt{Times} is overloaded by the * operator,
and is always meant to represent a commutative operator.
The other notable product is \texttt{Dot}, overloaded by the \texttt{.} operator.
This product represents matrix multiplication,
it is not commutative.
SymPy uses the same node for both scalar and matrix multiplication,
the only exception being with abstract matrix symbols.
Unlike Mathematica, SymPy determines commutativity with respect to
multiplication from the factor's expression type.
Mathematica puts the \texttt{Orderless} attribute on the expression
type.

% Associative expressions.

Regarding associative expressions,
SymPy handles associativity by making associative expressions inherit the
class \texttt{AssocOp},
while Mathematica specifies the \texttt{Flat} attribute on the expression type.

% Pattern matching


%% TODO list:
% * comparison with Mathematica: MatrixExp, product not always commutative, type inheritance (polymorphism) and advantage in unifying the product symbol * for symbols and matrices, pattern matching vs. single dispatch.
% * comparison with Mathematica: commutativity, associative expressions, one-identity. Advantage of SymPy: multiplicative commutativity defined on symbols.
% * comparison with Mathematica: avoid misspelling variables through forced declaration (check that you can't do it in Mathematica).
% * evaluate=False vs HoldForm
% * comparison with Mathematica: == is structural equality, not
% * comparison with Mathematica: polynomial module.
% * comparison with Mathematica: space is product, ** vs ^
% * comparison with Mathematica: ( ) is Sequence, functions are generally uppercase.
% * comparsion with Mathematica: table of comparison?
% * comparison with Mathematica: Wolfram language has loads of operator overloading, functional paradigm.


\section{Conclusion and future work}

\input{conclusion_and_future_work}

\section{References}

\bibliographystyle{siamplain}
\bibliography{paper}

\end{document}
