% SIAM Article Template
\documentclass[review]{siamart0216}
\usepackage{hyperref}
\usepackage{graphicx}
\usepackage{amsmath}
\usepackage{caption}
\graphicspath{ {images/} }

\usepackage{amsmath}
\usepackage{url}
\usepackage{hyperref}

% for nice source code syntax highlighting, also provides Listing env
\usepackage{minted}

% this is required for all the \url{} commands in the bib file
%\usepackage{hyperref}

% for nice units
\usepackage{siunitx}

% for images: png, pdf, etc
\usepackage{graphicx}

% for nice table formatting, i.e., /toprule, /midrule, etc
\usepackage{booktabs}

% to allow for \verb++ declarations in captions.
\usepackage{cprotect}

% to allow usage of \mathbb symbols
\usepackage{amssymb}

\usepackage{longtable}

\title{SymPy: Symbolic Computing in Python}

\input{authors}

\begin{document}
\maketitle

\begin{abstract}
  Computer algebra systems (CAS) play an important role in scientific computing
  with the earliest packages dating back into the 1960s.
  Numerous open and proprietary software packages exist that aid users in
  complex algebra, calculus, and other mathematical concepts that drive and
  support both symbolic and numerical computing needs.
  Here we present a CAS written in the Python programming language that is
  developed by a large team of scientists, engineers, and software developers
  in a bazaar-style open source development model.
  The accessibility of the codebase and the community model allow SymPy to
  rapidly respond to the needs of the community of users without the knowledge
  or consent of the original developers.
  This paper gives an overview of what SymPy is, what it offers with respect to
  similar software, and how it fits into the scientific python ecosystem.
  SymPy's core architecture is highlighted to show how it provides a
  foundation for good design patterns and efficient, readable code.
  To close, we present why and how SymPy is useful to researchers, educators,
  and the scientific software industry with examples of its use in these
  fields.
  The supplementary materials further outline the fine details of the
  architecture, various design decisions, and algorithms that make SymPy what
  it is today.
\end{abstract}

\section{Introduction}

%% What sympy is, where to download etc.
%%
%% List other major CASs.
%%
%% Why SymPy.

\input{introduction.tex}

\section{Architecture}

\input{architecture}

\section{Algorithms}

%% Description of some algorithms (example: integration with Risch, Meijer G, Gruntz, polys)
%%
%% Description of numerics/mpmath (Fredrik)

\input{algorithms}

\section{Features}

%% List of Features and how to use
%%
%% Quick overview of the main modules, what it can do and so on. It should probably provide examples how to use sympy.
%%
%% See also the supplement (below)

\input{features}

\input{domain_specific}

\section{Other Projects that use SymPy}

\input{other_projects_that_use_sympy}

\section{Comparison with other CAS}

\input{comparison_with_mma}

\section{Conclusion and future work}

\input{conclusion_and_future_work}

\section{References}

\bibliographystyle{siamplain}
\bibliography{paper}

\end{document}
