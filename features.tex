% Features to discuss in-depth:

SymPy has an extensive feature set that encompasses too much to cover
in-depth here. Bedrock areas, such a Calculus, receive their own sub-sections
below. Additionally, Table~\ref{features-table} describes other capabilities
present in the SymPy code base. This gives a sampling from the breadth of
topics and application domains that SymPy services.


\begin{longtable}[htbc]{|l|p{0.7\linewidth}|}
\caption{SymPy Features and Descriptions\label{features-table}}\\
\hline
\textbf{Feature} & \textbf{Description} \\
\hline
Discrete Math & Summations, products, binomial coefficients,
    prime number tools, integer factorization, Diophantine equation solving, and
    boolean logic representation, equivalence testing, and inference.\\
Concrete Math & Tools for determining whether summation and product
    expressions are convergent, absolutely convergent, hypergeometric, and
    other properties. May also compute Gosper's normal form~\cite{petkovvsek1996bak} for two univariate polynomials.\\
Plotting & Hooks for visualizing expressions via matplotlib~\cite{Hunter:2007}
    or as text drawings when lacking a graphical back-end.\\
Geometry & Allows the creation of 2D geometrical entities,
    such as lines and circles. Enables queries on these entities, including
    asking the area of an ellipse, checking for collinearity of a set of
    points, or finding the intersection between two lines.\\
Statistics & Support for a random variable type as well as the ability to
    declare this variable from prebuilt distribution functions such as
    Normal, Exponential, Coin, Die, and other custom distributions.\\
Polynomials & Computes polynomial algebras over various coefficient domains
    ranging from the simple (e.g., polynomial division) to the advanced
    (e.g., Gr\"obner bases~\cite{adams1994introduction} and multivariate
    factorization over algebraic number domains).\\
Sets & Representations of empty, finite, and infinite sets. This includes
    special sets such as for all natural, integer, and complex numbers.\\
Series & Implements series expansion, sequences, and limit of sequences.
    This includes special series, such as Fourier and power series.\\
Vectors & Provides basic vector math and differential calculus with respect
    to 3D Cartesian coordinate systems.\\
Matrices & Tools for creating matrices of symbols and expressions.
    This is capable of both sparse and dense representations and performing
    symbolic linear algebraic operations (e.g., inversion and factorization).\\
Combinatorics \& Group Theory & Implements permutations, combinations,
    partitions, subsets,
    various permutation groups (such as polyhedral, Rubik, symmetric,
    and others), Gray codes~\cite{Nijenhuis1978combinatorial},
    and Prufer sequences~\cite{biggs1976graph}.\\
Code Generation & Enables generation of compilable and executable
    code in a variety of different programming languages directly from
    expressions. Target languages include C, Fortran, Julia, JavaScript,
    Mathematica, Matlab and Octave, Python, and Theano.\\
Tensors & Symbolic manipulation of indexed objects.\\
Lie Algebras & Represents Lie algebras and root systems.\\
Cryptography & Represents block and stream ciphers, including
    shift, Affine, substitution, Vigenere's, Hill's, bifid, RSA, Kid RSA,
    linear-feedback shift registers, and Elgamal encryption\\
Special Functions & Implements a number of well known special functions,
    including Dirac delta, Gamma, Beta, Gauss error functions, Fresnel
    integrals, Exponential integrals, Logarithmic integrals, Trigonometric
    integrals, Bessel, Hankel, Airy, B-spline, Riemann Zeta, Dirichlet eta,
    polylogarithm, Lerch transcendent, hypergeometric, elliptic integrals,
    Mathieu, Jacobi polynomials, Gegenbauer polynomial, Chebyshev polynomial,
    Legendre polynomial, Hermite polynomial, Laguerre polynomial, and
    spherical harmonic functions.\\
\hline

\end{longtable}


% Basic operations (the core)
\subsection{Basic Operations}

\input{features_basic_operations}

\subsection{Calculus}

% Sets
\subsection{Sets}
%% Sets

SymPy supports representation of a wide variety of sets, this is achieved by
first defining abstract representation for a smaller number of atomic set
classes and then combining and transforming them using various set operations.

Each of the set classes inherits from the base set class and defines rules to
check membership of a SymPy object in that set, to calculate union,
intersection and set difference. In cases we are not able to evaluate these
operations to atomic set classes they are represented as abstract unevaluated
objects.


We have the following atomic set classes in SymPy.

\begin{itemize}

    \item \verb|EmptySet|: represents the empty set $\emptyset$.

    \item \verb|UniversalSet|: Everything is a member of Universal Set.
        Union of Universal Set with any set gives Universal Set and
        intersection leads to the other set itself.

    \item \verb|FiniteSet| is functionally equivalent to python's set
        object. Its members can be any SymPy object including other sets
        themselves.

    \item \verb|Integers| represents set of Integers $\mathbb{Z}$.

    \item \verb|Naturals| represents set of Natural numbers $\mathbb{N}$ i.e.,
        set of positive integers.

    \item \verb|Naturals0| represents the whole numbers which are all the
        non-negative integers, inclusive of zero.

    \item \verb|Range| represents a range of integers and is defined by
        specifying a start value, an end value and a step size. Range is
        functionally equivalent to python's range except the fact that it
        accepts infinity at end points allowing us to represent infinite
        ranges.


    \item \verb|RealInterval| is specified by giving the start and end point
        and specifying if it is open or closed in the respective ends. The set
        of real numbers is represented as a special case of a real interval
        where the start point is negative infinite and the end point is
        positive infinite.


\end{itemize}


%% Operations

Other than unevaluated classes of Union, Intersection and Set Difference
operations, we have following set classes.

\begin{itemize}

    \item \verb|ProductSet| abstractly defines the Cartesian product of two
        or more sets. Product Set is useful when representing higher
        dimensional spaces. For example to represent a three dimensional space
        we simply take the Cartesian product of three Real sets.

    \item \verb|ImageSet| represents the image of a function when applied to a
        particular set. In notation Image Set of a function $F$ w.r.t a set $S$
        is $\{ F(x) | x \in S \}$ In particular we use Image Set to represent
        the set of infinite solutions from trigonometric equations.


    \item \verb|ConditionSet| represents subset of a set who's members
        satisfies a particular condition. In notation Condition Set of set $S$
        w.r.t to a condition $H$ is $\{x | H(x), x \in S \}$. We use Condition Set
        to represent the set of solutions of an equation or an inequality where
        the equation or the inequality is the condition and the set is the
        domain in which we aim to find the solution.


\end{itemize}

A few other classes are implemented as special cases of the classes described
above. The real number \verb|Reals| is implemented as a special case of real
interval where the start point is negative infinity and the end point is
positive infinity. \verb|ComplexRegion| is implemented as a special case of
\verb|ImageSet|, \verb|ComplexRegion| supports both polar and rectangular
representation of region on the complex plane.


% Solvers (regular equations, maybe also mention other types of solvers like ODEs/recurrence/Diophantine)
\subsection{Solvers}
%% Solvers in SymPy


SymPy has module of equation solvers for symbolic equations. There are two
submodules to solve algebraic equations in SymPy, referred to as old solve
function, \texttt{solve}, and new solve function, \texttt{solveset}.
Solveset is introduced with several design changes with respect to old
\texttt{solve} function to resolve the issues with old \texttt{solve} function,
for example old \texttt{solve} function's input API has many flags which are
not needed and they make it hard for the user and the developers to work on
solvers. In contrast to old solve function, the \texttt{solveset} has a clean
input API, It only asks for the much needed information from the user, following
are the function signatures of old and new solve function:

\begin{verbatim}
solve(f, *symbols, **flags)  # old solve function
solveset(f, symbol, domain)  # new solve function
\end{verbatim}

The old \texttt{solve} function has an inconsistent output API for various types
of inputs, whereas the \texttt{solveset} has a canonical output API which is
achieved using sets. It can consistently return various types of solutions.

\begin{itemize}
\item Single solution
\end{itemize}
\begin{verbatim}
>>> solveset(x - 1)
>>> {1}
\end{verbatim}

\begin{itemize}
\item Finite set of solution, quadratic equation
\end{itemize}
\begin{verbatim}
>>> solveset(x**2 - pi**2, x)
{-pi, pi}
\end{verbatim}

\begin{itemize}
\item No Solution
\end{itemize}
\begin{verbatim}
>>> solveset(1, x)
EmptySet()
\end{verbatim}

\begin{itemize}
\item Interval of solution
\end{itemize}
\begin{verbatim}
>>> solveset(x**2 - 3 > 0, x, domain=S.Reals)
(-oo, -sqrt(3)) U (sqrt(3), oo)
\end{verbatim}

\begin{itemize}
\item Infinitely many solutions
\end{itemize}
\begin{verbatim}
>>> solveset(sin(x) - 1, x, domain=S.Reals)
ImageSet(Lambda(_n, 2*_n*pi + pi/2), Integers())
>>> solveset(x - x, x, domain=S.Reals)
(-oo, oo)
>>> solveset(x - x, x, domain=S.Complexes)
S.Complexes
\end{verbatim}

\begin{itemize}
\item Linear system: finite and infinite solution for determined, under
determined and over determined problems.
\end{itemize}
\begin{verbatim}
>>> A = Matrix([[1, 2, 3], [4, 5, 6], [7, 8, 10]])
>>> b = Matrix([3, 6, 9])
>>> linsolve((A, b), x, y, z)
{(−1,2,0)}
>>> linsolve(Matrix(([1, 1, 1, 1], [1, 1, 2, 3])), (x, y, z))
{(-y - 1, y, 2)}
\end{verbatim}

The new solve i.e. \textbf{solveset} is under active development and is a
planned replacement for \textbf{solve}, Hence there are some features which are
implemented in solve and is not yet implemented in solveset. The table below
show the current state of old and new solve functions.

\hfill

\begin{tabular}{ |p{4cm}|p{3cm}|p{3cm}|  }
\hline
\multicolumn{3}{|c|}{Solveset vs Solve} \\
\hline
Feature& solve &solveset \\
\hline
Consistent Output API & No & Yes \\
Consistent Input API & No & Yes \\
Univariate & Yes & Yes\\
Linear System& Yes & Yes (linsolve) \\
Non Linear System& Yes & Not yet \\
Transcendental& Yes & Not yet \\
\hline
\end{tabular}

\hfill \break{}

Below are some of the examples of old \textbf{solve} function:

\begin{itemize}
\item Non Linear (multivariate) System of Equation: Intersection of a circle
and a parabola.
\end{itemize}
\begin{verbatim}
>>> solve([x**2 + y**2 - 16, 4*x - y**2 + 6], x, y)
[(-2 + sqrt(14), -sqrt(-2 + 4*sqrt(14))),
 (-2 + sqrt(14), sqrt(-2 + 4*sqrt(14))),
 (-sqrt(14) - 2, -I*sqrt(2 + 4*sqrt(14))),
 (-sqrt(14) - 2, I*sqrt(2 + 4*sqrt(14)))]
\end{verbatim}

\begin{itemize}
\item Transcendental Equation
\end{itemize}
\begin{verbatim}
>>> solve(x + log(x))**2 - 5*(x + log(x)) + 6, x)
[LambertW(exp(2)), LambertW(exp(3))]
>>> solve(x**3 + exp(x))
[-3*LambertW((-1)**(2/3)/3)]
\end{verbatim}


Diophantine equations play a central and an important role in number theory.
A Diophantine equation has the form, $f(x_1, x_2, \ldots x_n) = 0$
where $n \geq 2$ and $x_1, x_2, \ldots x_n$ are integer variables. If we can find
$n$ integers $a_1, a_2, \ldots a_n$ such that $x_1 = a_1, x_2 = a_2, \ldots x_n = a_n$
satisfies the above equation, we say that the equation is solvable.

Currently, following five types of Diophantine equations can be solved using
SymPy's Diophantine module.

\begin{itemize}
    \item Linear Diophantine equations: $a_1x_1 + a_2x_2 + \cdots + a_{n}x_{n} = b$
    \item General binary quadratic equation: $ax^2 + bxy + cy^2 + dx + ey + f = 0$
    \item Homogeneous ternary quadratic equation: $ax^2 + by^2 + cz^2 + dxy + eyz + fzx = 0$
    \item Extended Pythagorean equation: $a_{1}x_{1}^2 + a_{2}x_{2}^2 + \cdots + a_{n}x_{n}^2 = a_{n+1}x_{n+1}^2$
    \item General sum of squares: $x_{1}^2 + x_{2}^2 + \cdots + x_{n}^2 = k$
\end{itemize}

When an equation is fed into Diophantine module, it factors the equation (if
possible) and solves each factor separately. Then all the results are combined to
create the final solution set. Following examples illustrate some of the basic
functionalities of the Diophantine module.

\begin{verbatim}
>>> from sympy import symbols
>>> x, y, z = symbols("x, y, z", integer=True)

>>> diophantine(2*x + 3*y - 5)
set([(3*t_0 - 5, -2*t_0 + 5)])

>>> diophantine(2*x + 4*y - 3)
set()

>>> diophantine(x**2 - 4*x*y + 8*y**2 - 3*x + 7*y - 5)
set([(2, 1), (5, 1)])

>>> diophantine(x**2 - 4*x*y + 4*y**2 - 3*x + 7*y - 5)
set([(-2*t**2 - 7*t + 10, -t**2 - 3*t + 5)])

>>> diophantine(3*x**2 + 4*y**2 - 5*z**2 + 4*x*y - 7*y*z + 7*z*x)
set([(-16*p**2 + 28*p*q + 20*q**2, 3*p**2 + 38*p*q - 25*q**2, 4*p**2 - 24*p*q + 68*q**2)])

>>> from sympy.abc import a, b, c, d, e, f
>>> diophantine(9*a**2 + 16*b**2 + c**2 + 49*d**2 + 4*e**2 - 25*f**2)
set([(70*t1**2 + 70*t2**2 + 70*t3**2 + 70*t4**2 - 70*t5**2, 105*t1*t5, 420*t2*t5, 60*t3*t5, 210*t4*t5, 42*t1**2 + 42*t2**2 + 42*t3**2 + 42*t4**2 + 42*t5**2)])

>>> diophantine(a**2 + b**2 + c**2 + d**2 + e**2 + f**2 - 112)
set([(8, 4, 4, 4, 0, 0)])
\end{verbatim}


% Matrices (worth including to stress that they are symbolic)
\subsection{Matrices}

\input{features_matrices}

% Physics module (some sampling, to show that it is there)
\subsection{Physics}

% Logic module
\subsection{Logic}

SymPy supports construction and manipulation of boolean expressions
through the \texttt{logic} module. SymPy symbols can be used as
propositional variables and also be substituted as \texttt{True}
or \texttt{False}. A good number of manipulation features for boolean
expressions have been implemented in the \texttt{logic} module.

\subsubsection{Constructing boolean expressions}

A boolean variable can be declared as a SymPy symbol. Python
operators \&, \textbar{} and \textasciitilde{} are overloaded for logical \texttt{And},
\texttt{Or} and \texttt{negate}. Several others like \texttt{Xor},
\texttt{Implies} can be constructed with \^{}, \textgreater\textgreater{} respectively.
The above are just a shorthand, expressions can also be constructed
by directly calling \verb|And()|, \verb|Or()|, \verb|Not()|,
\verb|Xor()|, \verb|Nand()|, \verb|Nor()|, etc.

\begin{verbatim}
>>> from sympy import *
>>> x, y, z = symbols('x y z')
>>> e = (x & y) | z
>>> e.subs({x: True, y: True, z: False})
True
\end{verbatim}

\subsubsection{CNF and DNF}

Any boolean expression can be converted to conjunctive normal
form, disjunctive normal form and negation normal form. The
API also permits to check if a boolean expression is in any
of the above mentioned forms.

\begin{verbatim}
>>> from sympy import *
>>> x, y, z = symbols('x y z')
>>> to_cnf((x & y) | z)
And(Or(x, z), Or(y, z))
>>> to_dnf(x & (y | z))
Or(And(x, y), And(x, z))
>>> is_cnf((x | y) & z)
True
>>> is_dnf((x & y) | z)
True
\end{verbatim}

\subsubsection{Simplification and Equivalence}

The module supports simplification of given boolean expression
by making deductions on it. Equivalence of two expressions can
also be checked. If so, it is possible to return the mapping of
variables of two expressions so as to represent the
same logical behaviour.

\begin{verbatim}
>>> from sympy import *
>>> a, b, c, x, y, z = symbols('a b c x y z')
>>> e = a & (~a | ~b) & (a | c)
>>> simplify(e)
And(Not(b), a)
>>> e1 = a & (b | c)
>>> e2 = (x & y) | (x & z)
>>> bool_map(e1, e2)
(And(Or(b, c), a), {b: y, a: x, c: z})
\end{verbatim}

\subsubsection{SAT solving}

The module also supports satisfiability checking of a given
boolean expression. If satisfiable, it is possible to return
a model for which the expression is satisfiable. The API also
supports returning all possible models. The SAT solver has
a clause learning DPLL algorithm implemented with watch
literal scheme and VSIDS heuristic\cite{moskewicz2008method}.

\begin{verbatim}
>>> from sympy import *
>>> a, b, c = symbols('a b c')
>>> satisfiable(a & (~a | b) & (~b | c) & ~c)
False
>>> satisfiable(a & (~a | b) & (~b | c) & c)
{b: True, a: True, c: True}
\end{verbatim}

