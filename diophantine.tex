Diophantine equations play a central and an important role in number theory.
A Diophantine equation has the form, $f(x_1, x_2, \ldots x_n) = 0$
where $n \geq 2$ and $x_1, x_2, \ldots x_n$ are integer variables. If we can find
$n$ integers $a_1, a_2, \ldots a_n$ such that $x_1 = a_1, x_2 = a_2, \ldots x_n = a_n$
satisfies the above equation, we say that the equation is solvable.

Currently, following five types of Diophantine equations can be solved using
SymPy's Diophantine module.

\begin{itemize}
    \item Linear Diophantine equations: $a_1x_1 + a_2x_2 + \cdots + a_{n}x_{n} = b$
    \item General binary quadratic equation: $ax^2 + bxy + cy^2 + dx + ey + f = 0$
    \item Homogeneous ternary quadratic equation: $ax^2 + by^2 + cz^2 + dxy + eyz + fzx = 0$
    \item Extended Pythagorean equation: $a_{1}x_{1}^2 + a_{2}x_{2}^2 + \cdots + a_{n}x_{n}^2 = a_{n+1}x_{n+1}^2$
    \item General sum of squares: $x_{1}^2 + x_{2}^2 + \cdots + x_{n}^2 = k$
\end{itemize}

When an equation is fed into Diophantine module, it factors the equation (if
possible) and solves each factor separately. Then all the results are combined to
create the final solution set. Following examples illustrate some of the basic
functionalities of the Diophantine module.

\begin{verbatim}
>>> from sympy import symbols
>>> x, y, z = symbols("x, y, z", integer=True)

>>> diophantine(2*x + 3*y - 5)
set([(3*t_0 - 5, -2*t_0 + 5)])

>>> diophantine(2*x + 4*y - 3)
set()

>>> diophantine(x**2 - 4*x*y + 8*y**2 - 3*x + 7*y - 5)
set([(2, 1), (5, 1)])

>>> diophantine(x**2 - 4*x*y + 4*y**2 - 3*x + 7*y - 5)
set([(-2*t**2 - 7*t + 10, -t**2 - 3*t + 5)])

>>> diophantine(3*x**2 + 4*y**2 - 5*z**2 + 4*x*y - 7*y*z + 7*z*x)
set([(-16*p**2 + 28*p*q + 20*q**2, 3*p**2 + 38*p*q - 25*q**2, 4*p**2 - 24*p*q + 68*q**2)])

>>> from sympy.abc import a, b, c, d, e, f
>>> diophantine(9*a**2 + 16*b**2 + c**2 + 49*d**2 + 4*e**2 - 25*f**2)
set([(70*t1**2 + 70*t2**2 + 70*t3**2 + 70*t4**2 - 70*t5**2, 105*t1*t5, 420*t2*t5, 60*t3*t5, 210*t4*t5, 42*t1**2 + 42*t2**2 + 42*t3**2 + 42*t4**2 + 42*t5**2)])

>>> diophantine(a**2 + b**2 + c**2 + d**2 + e**2 + f**2 - 112)
set([(8, 4, 4, 4, 0, 0)])
\end{verbatim}
